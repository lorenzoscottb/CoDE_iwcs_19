\documentclass{article}
\usepackage[utf8]{inputenc}
\usepackage{natbib}
\usepackage{graphicx}
\usepackage{varwidth}


\title{All you need is GloVe: \\ generating composition-oriented syntax-aware embeddings from the APT framework using a GloVe-based model}
\author{L.S.Bertolini}
\date{March 2019}


\begin{document}

\maketitle

\section{Introduction}
Anchored Packed Trees (APTs) is a recently introduced count-based
distributional semantics approach, that relies on the syntactic
contextualization of lexeme. That is its distributional features are 
represented by (weighted) observed high-order 
dependency 
paths
\newline
\newline
\begin{varwidth}[t]{.5\textwidth}
\begin{description}
\item car:\{ $\textit{amod}$:red
\begin{description}
\item$\overline{dobj}$:drive,
\item$\textit{nmod}$:garage
\item$\vdots$
\item$\overline{nmod}$.$\textit{nsubj}$:traveller\} %'_nmod»nsubj:traveller'
\end{description}
\end{description}
\end{varwidth}% <---- Don't forget this %
\hspace{4em}% <---- Don't forget this %
\begin{varwidth}[t]{.5\textwidth}
\begin{description}
\item red:\{ $\overline{amod}$:car,
\begin{description}
\item$\overline{amod}$.$\overline{dobj}$:bought,
\item$\overline{amod}$.$\textit{det}$:the
\item$\vdots$
\item$\overline{nmod}$:wine\}
\end{description}
\end{description}
\end{varwidth}\newline
\newline
\newline
Where the overline on top of dependency indicates that, to reach that node, 
we need to travel such path in its inverse direction.\\
Given such formalization, every lexeme can be considered as a series of 
dependency trees, seen from the point of view -anchored- of that specific 
token. 

\section{Conclusion}
